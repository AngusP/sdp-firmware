%
%  FIRMWARE TECHNICAL SPECIFICATION
%

\section{Firmware}

\subsection{Architecture}

The firmware we're currently running has changed significantly from the source we started with (SDP Group 13 2015). Some useful concepts have survived, but the implementation is significantly different, and the majority of changes have been made to generalise the control structures and add powerful flexibility to the firmware's capabilities.

The notion of `Processes' has been introduced, allowing a structure to define a task that must be run at a defined time interval. Process structures carry around all their own scheduling information, and a pointer to the function that needs to be run; This information is mutable, allowing temporal state machine behaviour with time-triggered transitions to be employed. The firmware as it stands uses Processes to poll the rotary encoders on the drive motors, and calculate the instantaneous speeds of each wheel. A Process was also used to guarantee accurate timing for the Milestone 1 communications task. This architecture adds minimal overhead but verifiably gives an abstracted way of serially multitasking, as opposed to a very large loop function that risks control blocking and lock-ups.

Similarly, the parsing and execution of commands received over serial is handled by the `CommandSet' class and `SerialCommand' subclass, with function pointers being registered in association with the keyword string at setup time. Commands received over Serial are interpreted and executed immediately in an interrupt-driven model, giving very little lag between a command being sent and a command being enacted by the bot. These abstractions make it easy to add processes or new commands, and isolate potentially buggy functions from each-other (though any blocking IO operations will still break functionality). It is also possible to enable and disable processes during runtime, if it is desired. 

\subsection{Motion Execution and Correction}

A process (as defined above) is used to read the changes in rotary encoder positions from the three NXT motors. Another process then uses this data to correct to motor speeds to reflect the motion the robot was commanded to execute. The correction process uses Gradient Descent, which is necessary due to the alinear nature of the Motors' rotational speed with respect to power applied, coupled with the fact that each of the three motors has a different signature response.

The error vector $\hat{e}$ given a desired velocity vector $\hat{v}$ and realised velocity vector $\hat{r}$ is defined:

\begin{equation}
  \hat{e_i} = k \cdot (\frac{\hat{d_i}}{||\hat{d}||} - \frac{\hat{r_i}}{||\hat{r}||})
\end{equation}

The new powers we need to apply to the motors to reduce the error, $\hat{c'}$ with respect to the previous $\hat{c}$ and error $\hat{e}$ is

\begin{equation}
  \hat{c_i}' = \frac{ \hat{c_i} }{||\hat{c}||} + \frac{ \hat{e_i} }{||\hat{e}||}
\end{equation}

The error function we wish to minimise is $E$

\begin{equation}
  E_{(\hat{d},\hat{r})} = \sqrt{\sum_{j=1}^{n} (\hat{d_j} - \hat{r_j})^2}
\end{equation}

Which has the partial derivative with respect to $\hat{r_i}$

\begin{equation}
  \frac{\partial E}{\partial \hat{r_i}}
  =
  \frac{\hat{r_i} - \hat{d_i}}{\sqrt{\sum_{j=1}^{n} (\hat{d_j} - \hat{r_j})^2}}
  =
  \frac{\hat{r_i} - \hat{d_i}}{||\hat{d}-\hat{r}||}
\end{equation}

This component is crucial to the accuracy and repeatability of motion for the robot.

As far as units go, all vectors are converted to unit vectors, but most are provided initially with dimensions in the range ${\{-255,255\}}$ with the exception of the realised velocity vector, which is simply the number of ticks the rotary encoders have passed in the time interval since last checked. The conversion to unit vectors is in part to eliminate problems with unit compatibility, and part because the single-precision only floating point unit on the Arduino quickly succumbs to NaNs if dealing with larger numbers. 

